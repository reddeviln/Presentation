%% UzK - A BEAMER THEME FOR THE UNIVERSITY OF COLOGNE
%% http://solstice.github.com/uzk-theme/

\documentclass{beamer}
\usepackage[ngerman]{babel}
\usepackage[T1]{fontenc}
\usepackage[utf8]{inputenc}
\usepackage{graphicx}
\usepackage{amsmath} % Erweiterung für den Mathe-Satz
\usepackage{amssymb}
\usepackage{algorithm}
\usepackage{algorithmic}
\usepackage{multimedia}
\usepackage{tikz}
\usetikzlibrary{shapes,arrows}
%% Falls Anzeige der \sections, \subsections etc. gewuenscht, kann zB.
%% das infolines theme geladen werden. Wichtig ist jedoch, dass andere
%% Themes _vor_ dem UzK-Theme geladen werden.
%\useoutertheme{infolines}

%% Falls keine der Optionen zur Bestimmung der Fusszeile uebergeben werden    %%
%% werden alle Fakultaetsfarben verwendet. ---------------------------------- %%
\newcommand{\abs}[1]{\left\vert #1\right\vert}
\newcommand{\rr}{\mathbb{R}}
\newcommand{\g}{~\textgreater ~}
\newcommand{\ls}{~\textless ~}
\renewcommand{\algorithmicrequire}{\textbf{Input:}}
\renewcommand{\algorithmicensure}{\textbf{Output:}}
\newcommand{\cc}{\mathbb{C}}
\newcommand{\e}{\varepsilon\g 0~}
\newcommand{\fe}{\forall \e}
\newcommand{\so}{\sum_{k=0}^{n}}
\newcommand{\si}{\sum_{k=1}^{n}}
\newcommand{\soi}{\sum_{k=0}^{\infty}}
\newcommand{\sii}{\sum_{k=1}^{\infty}}
\newcommand{\de}{\mathrm{d}}
\newcommand{\nn}{\mathbb{N}}
\newcommand{\norm}[1]{\left\lVert#1\right\rVert}
\theoremstyle{definition}
\newtheorem{auf}{Aufgabe}
\newtheorem{rem}[auf]{Bemerkung}
\newtheorem{defn}[auf]{Definition}
\newtheorem{bsp}[auf]{Beispiel}
\theoremstyle{plain}
\newtheorem{kor}[auf]{Korollar}
\newtheorem{sa}[auf]{Satz}
\newtheorem{alg}[auf]{Algorithmus}
\newtheorem{lem}[auf]{Lemma}
\floatname{algorithm}{Algorithmus}
\usetheme[%
%wiso,        %% Wiso-Fakultaet
%jura,        %% Rechtswissenschaftliche Fakultaet
%medizin,     %% Medizinische Fakultaet
%philo,       %% Philosophische Fakultaet
matnat,      %% Mathematisch-Naturwissenschaftliche Fakultaet
%human,       %% Humanwissenschaftliche Fakultaet
%verw,        %% Universitaetsverwaltung
%nav,         %% Schaltet die Navigationssymbole ein
%latexfonts,  %% Verwendet die latex-beamer-Standardschrift
%colorful,    %% Farbige Balken im infolines-Theme
%squares,     %% Aufzaehlungspunkte rechteckig
%nologo,      %% Kein Logo im Seitenhintergrund
]{UzK}

\title{Das \textsc{Györi-Lovasz}-Theorem}
\author{Nils Dornbusch}%
\subtitle{Seminarvortrag}
\institute{ }

\begin{document}
\begin{frame}
  \titlepage
\end{frame}

\begin{frame}
\frametitle{Das \textsc{Györi-Lovasz}-Theorem}
\begin{sa}[Satz 1]
Sei $k\ge 2$ und $G$ ein $k$-zusammenhängender Graph mit $n$ Knoten. Seien $v_1,\dotsc,v_k$ paarweise verschiedene Knoten und $n_1,\dotsc,n_k\in\nn$ mit $\sum_{i=1}^kn_i=n$. Dann hat $G$ disjunkte zusammenhängende Teilgraphen $G_1,\dotsc,G_k$, sodass für $i=1,\dotsc,k$ $G_i$ genau $n_i$ Knoten hat und $v_i\in V(G_i)$ ist.
\end{sa}
\end{frame}
\begin{frame}
\frametitle{Beispiel zum Theorem}
Hier für $k=2$ und $n_1=2=n_2$:
\begin{figure}
\begin{tikzpicture}{
    \node[draw, circle] (v1) at (0,2) {$v_1$};
    \node[draw, circle] (v2) at (2,0) {$v_2$};
    \node[draw, circle] (v3) at (0,-2) {$v_3$};
    \node[draw, circle] (v4) at (-2,0) {$v_4$};
    \draw (v1) edge[-] (v2) (v2) edge[-] (v3) (v3) edge[-] (v4) (v4) edge[-] (v1);\pause
    \draw[rotate=45] (0,1.35) ellipse (2.2 and 1);\pause
    \draw[rotate=45] (0,-1.35) ellipse (2.2 and 1);\pause
    \node (note) at (-2,2) {$G_1$};
    \node (note2) at (2,-2) {$G_2$};
}
\end{tikzpicture}
\end{figure}
\end{frame}
\begin{frame}
\frametitle{Beispiel zum Theorem (2)}
Für $k=2$ und $n_1=3,~n_2=1$:
\begin{figure}
\begin{tikzpicture}{
    \node[draw, circle] (v1) at (0,2) {$v_1$};
    \node[draw, circle] (v2) at (2,0) {$v_2$};
    \node[draw, circle] (v3) at (0,-2) {$v_3$};
    \node[draw, circle] (v4) at (-2,0) {$v_4$};
    \draw (v1) edge[-] (v2) (v2) edge[-] (v3) (v3) edge[-] (v4) (v4) edge[-] (v1);\pause
    \draw plot [smooth cycle] coordinates {(0,0) (0.8,-2) (0,-2.8) (-2.6,0) (0,2.8) (0.8,2.3)};\pause
    \draw (v2) circle (0.7);\pause
    \node (note) at (-2,2) {$G_1$};
    \node (note2) at (3.3,0) {$G_2$};
}
\end{tikzpicture}
\end{figure}
\end{frame}
\begin{frame}
\frametitle{Den Satz den wir beweisen werden...}
\begin{sa}[Satz 2]
Sei $k\ge 2$ und $G$ ein $k$-zusammenhängender Graph mit $n$ Knoten. Seien $v_1,\dotsc,v_k$ paarweise verschiedene Knoten und $n_1,\dotsc,n_k\in\nn$ mit $\sum_{i=1}^kn_i<n$. Seien $G_1,\dotsc,G_k$ disjunkte zusammenhängende Teilgraphen , sodass für $i=1,\dotsc,k$ $G_i$ genau $n_i$ Knoten hat und $v_i\in V(G_i)$ ist. Dann hat $G$ disjunkte zusammenhängende Teilgraphen $G'_1,\dotsc,G'_k$, sodass $v_i\in V(G'_i)~\forall i=1,\dotsc,k$. $G'_1$ hat $n_1+1$ Knoten und alle anderen $G'_i$ $n_i$ Knoten.
\end{sa}
\end{frame}
\begin{frame}
\frametitle{Vorgehensweise}
So wollen wir vorgehen:
\begin{itemize}
\item  Führen einige Begriffe aus der Hydrologie ein (Verständnis)\pause
\item Beweisen von Lemmata\pause
\item Beweisen von Satz 2 durch Widerspruch\pause
\item daraus folgt das \textsc{Györi-Lovasz}-Theorem
\end{itemize}
\end{frame}
\begin{frame}
\frametitle{Reservoir}
\label{Res}
\only<1>{\begin{defn}
Seien $G_1,\dotsc, G_k$ wie in Satz 2 und $i=2,\dotsc,k$. Für einen Knoten $v\in V\setminus G_i$ definieren wir ein \emph{Reservoir} 
\[R(v)=\left\lbrace u\in V(G_i)\colon \exists u-v_i-\text{Weg in }G_i\setminus v\right\rbrace\] 
\end{defn}}
\only<2->{
$R(v_3)=\lbrace v_1,v_2\rbrace$, $R(v_1)=\lbrace v_3\rbrace$ und $R(v_2)=\lbrace v_3, v_1\rbrace$ Man sieht also $v\not\in R(v)$  und $R(v_i)=\emptyset$.
\begin{figure}
 \begin{tikzpicture}
        \node[draw, circle] (v1) at (0,-2) {$v_1$};
        \node[draw, circle] (v2) at (2,0) {$v_2$};
        \node[draw, circle] (v3) at (-2,0) {$v_3$};
        \node[draw, circle] (vi) at (0,2) {$v_i$};
        \draw (v1) edge[-] (v2)  (v3) edge[-] (vi) (vi) edge[-] (v1);
    \end{tikzpicture}
\end{figure}}
\end{frame}
\begin{frame}
\frametitle{Kaskade}
\begin{defn}
Eine \emph{Kaskade} in $G_i$ ist eine (möglicherweise leere) Folge $w_1,\dotsc,w_m$ von Knoten in $G_i\setminus v_i$, sodass $w_{j+1}\not\in R(w_j)~\forall j=1,\dotsc,m-1$ 
\end{defn}
\end{frame}
\begin{frame}
\frametitle{Beispiel einer Kaskade}
\begin{figure}
 \begin{tikzpicture}
        \node[draw, circle] (v1) at (0,-2) {$v_1$};
        \node[draw, circle] (v2) at (2,0) {$v_2$};
        \node[draw, circle] (v3) at (-2,0) {$v_3$};
        \node[draw, circle] (vi) at (0,2) {$v_i$};
        \draw (v1) edge[-] (v2)  (v3) edge[-] (vi) (vi) edge[-] (v1);
    \end{tikzpicture}
\end{figure}
Mögliche Kaskaden:
\begin{enumerate}
    \item die leere triviale Kaskade
    \item die Folge: $v_1 v_2$
\end{enumerate}
\end{frame}
\end{document}