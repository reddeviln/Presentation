%% UzK - A BEAMER THEME FOR THE UNIVERSITY OF COLOGNE
%% http://solstice.github.com/uzk-theme/

\documentclass{beamer}
\usepackage[ngerman]{babel}
\usepackage[T1]{fontenc}
\usepackage[utf8]{inputenc}
\usepackage{graphicx}
\usepackage{amsmath} % Erweiterung für den Mathe-Satz
\usepackage{amssymb}
\usepackage{algorithm}
\usepackage{algorithmic}
\usepackage{multimedia}
\usepackage{tikz}
\usetikzlibrary{shapes,arrows}
%% Falls Anzeige der \sections, \subsections etc. gewuenscht, kann zB.
%% das infolines theme geladen werden. Wichtig ist jedoch, dass andere
%% Themes _vor_ dem UzK-Theme geladen werden.
%\useoutertheme{infolines}

%% Falls keine der Optionen zur Bestimmung der Fusszeile uebergeben werden    %%
%% werden alle Fakultaetsfarben verwendet. ---------------------------------- %%
\newcommand{\abs}[1]{\left\vert #1\right\vert}
\newcommand{\rr}{\mathbb{R}}
\newcommand{\g}{~\textgreater ~}
\newcommand{\ls}{~\textless ~}
\renewcommand{\algorithmicrequire}{\textbf{Input:}}
\renewcommand{\algorithmicensure}{\textbf{Output:}}
\newcommand{\cc}{\mathbb{C}}
\newcommand{\e}{\varepsilon\g 0~}
\newcommand{\fe}{\forall \e}
\newcommand{\so}{\sum_{k=0}^{n}}
\newcommand{\si}{\sum_{k=1}^{n}}
\newcommand{\soi}{\sum_{k=0}^{\infty}}
\newcommand{\sii}{\sum_{k=1}^{\infty}}
\newcommand{\de}{\mathrm{d}}
\newcommand{\nn}{\mathbb{N}}
\newcommand{\norm}[1]{\left\lVert#1\right\rVert}
\theoremstyle{definition}
\newtheorem{auf}{Aufgabe}
\newtheorem{rem}[auf]{Bemerkung}
\newtheorem{defn}[auf]{Definition}
\newtheorem{bsp}[auf]{Beispiel}
\theoremstyle{plain}
\newtheorem{kor}[auf]{Korollar}
\newtheorem{sa}[auf]{Satz}
\newtheorem{alg}[auf]{Algorithmus}
\newtheorem{lem}[auf]{Lemma}
\floatname{algorithm}{Algorithmus}
\usetheme[%
%wiso,        %% Wiso-Fakultaet
%jura,        %% Rechtswissenschaftliche Fakultaet
%medizin,     %% Medizinische Fakultaet
%philo,       %% Philosophische Fakultaet
matnat,      %% Mathematisch-Naturwissenschaftliche Fakultaet
%human,       %% Humanwissenschaftliche Fakultaet
%verw,        %% Universitaetsverwaltung
%nav,         %% Schaltet die Navigationssymbole ein
%latexfonts,  %% Verwendet die latex-beamer-Standardschrift
%colorful,    %% Farbige Balken im infolines-Theme
%squares,     %% Aufzaehlungspunkte rechteckig
%nologo,      %% Kein Logo im Seitenhintergrund
]{UzK}

\title{Das \textsc{Györi-Lovasz}-Theorem}
\author{Nils Dornbusch}%
\subtitle{Seminarvortrag}
\institute{ }

\begin{document}
\begin{frame}
  \titlepage
\end{frame}

\begin{frame}
\frametitle{Das \textsc{Györi-Lovasz}-Theorem}
\begin{sa}[Satz 1]
Sei $k\ge 2$ und $G$ ein $k$-zusammenhängender Graph mit $n$ Knoten. Seien $v_1,\dotsc,v_k$ paarweise verschiedene Knoten und $n_1,\dotsc,n_k\in\nn$ mit $\sum_{i=1}^kn_i=n$. Dann hat $G$ disjunkte zusammenhängende Teilgraphen $G_1,\dotsc,G_k$, sodass für $i=1,\dotsc,k$ $G_i$ genau $n_i$ Knoten hat und $v_i\in V(G_i)$ ist.
\end{sa}
\end{frame}
\begin{frame}
\frametitle{Beispiel zum Theorem}
Hier für $k=2$ und $n_1=2=n_2$:
\begin{figure}
\begin{tikzpicture}{
    \node[draw, circle] (v1) at (0,2) {$v_1$};
    \node[draw, circle] (v2) at (2,0) {$v_2$};
    \node[draw, circle] (v3) at (0,-2) {$v_3$};
    \node[draw, circle] (v4) at (-2,0) {$v_4$};
    \draw (v1) edge[-] (v2) (v2) edge[-] (v3) (v3) edge[-] (v4) (v4) edge[-] (v1);\pause
    \draw[rotate=45] (0,1.35) ellipse (2.2 and 1);\pause
    \draw[rotate=45] (0,-1.35) ellipse (2.2 and 1);\pause
    \node (note) at (-2,2) {$G_1$};
    \node (note2) at (2,-2) {$G_2$};
}
\end{tikzpicture}
\end{figure}
\end{frame}
\begin{frame}
\frametitle{Beispiel zum Theorem (2)}
Für $k=2$ und $n_1=3,~n_2=1$:
\begin{figure}
\begin{tikzpicture}{
    \node[draw, circle] (v1) at (0,2) {$v_1$};
    \node[draw, circle] (v2) at (2,0) {$v_2$};
    \node[draw, circle] (v3) at (0,-2) {$v_3$};
    \node[draw, circle] (v4) at (-2,0) {$v_4$};
    \draw (v1) edge[-] (v2) (v2) edge[-] (v3) (v3) edge[-] (v4) (v4) edge[-] (v1);\pause
    \draw plot [smooth cycle] coordinates {(0,0) (0.8,-2) (0,-2.8) (-2.6,0) (0,2.8) (0.8,2.3)};\pause
    \draw (v2) circle (0.7);\pause
    \node (note) at (-2,2) {$G_1$};
    \node (note2) at (3.3,0) {$G_2$};
}
\end{tikzpicture}
\end{figure}
\end{frame}
\begin{frame}
\frametitle{Den Satz den wir beweisen werden...}
\begin{sa}[Satz 2]
Sei $k\ge 2$ und $G$ ein $k$-zusammenhängender Graph mit $n$ Knoten. Seien $v_1,\dotsc,v_k$ paarweise verschiedene Knoten und $n_1,\dotsc,n_k\in\nn$ mit $\sum_{i=1}^kn_i<n$. Seien $G_1,\dotsc,G_k$ disjunkte zusammenhängende Teilgraphen , sodass für $i=1,\dotsc,k$ $G_i$ genau $n_i$ Knoten hat und $v_i\in V(G_i)$ ist. Dann hat $G$ disjunkte zusammenhängende Teilgraphen $G'_1,\dotsc,G'_k$, sodass $v_i\in V(G'_i)~\forall i=1,\dotsc,k$. $G'_1$ hat $n_1+1$ Knoten und alle anderen $G'_i$ $n_i$ Knoten.
\end{sa}
\end{frame}
\begin{frame}
\frametitle{Vorgehensweise}
So wollen wir vorgehen:
\begin{itemize}
\item  Führen einige Begriffe aus der Hydrologie ein (Verständnis)\pause
\item Beweisen von Lemmata\pause
\item Beweisen von Satz 2 durch Widerspruch\pause
\item daraus folgt das \textsc{Györi-Lovasz}-Theorem
\end{itemize}
\end{frame}
\begin{frame}
\frametitle{Reservoir}
\label{Res}
\only<1>{\begin{defn}
Seien $G_1,\dotsc, G_k$ wie in Satz 2 und $i=2,\dotsc,k$. Für einen Knoten $v\in V\setminus G_i$ definieren wir ein \emph{Reservoir} 
\[R(v)=\left\lbrace u\in V(G_i)\colon \exists u-v_i-\text{Weg in }G_i\setminus v\right\rbrace\] 
\end{defn}}
\only<2->{
$R(v_3)=\lbrace v_1,v_2\rbrace$, $R(v_1)=\lbrace v_3\rbrace$ und $R(v_2)=\lbrace v_3, v_1\rbrace$ Man sieht also $v\not\in R(v)$  und $R(v_i)=\emptyset$.
\begin{figure}
 \begin{tikzpicture}
        \node[draw, circle] (v1) at (0,-2) {$v_1$};
        \node[draw, circle] (v2) at (2,0) {$v_2$};
        \node[draw, circle] (v3) at (-2,0) {$v_3$};
        \node[draw, circle] (vi) at (0,2) {$v_i$};
        \draw (v1) edge[-] (v2)  (v3) edge[-] (vi) (vi) edge[-] (v1);
    \end{tikzpicture}
\end{figure}}
\end{frame}
\begin{frame}
\frametitle{Kaskade}
\begin{defn}
Eine \emph{Kaskade} in $G_i$ ist eine (möglicherweise leere) Folge $w_1,\dotsc,w_m$ von Knoten in $G_i\setminus v_i$, sodass $w_{j+1}\not\in R(w_j)~\forall j=1,\dotsc,m-1$ 
\end{defn}
\end{frame}
\begin{frame}
\frametitle{Beispiel einer Kaskade}
\begin{figure}
 \begin{tikzpicture}
        \node[draw, circle] (v1) at (0,-2) {$v_1$};
        \node[draw, circle] (v2) at (2,0) {$v_2$};
        \node[draw, circle] (v3) at (-2,0) {$v_3$};
        \node[draw, circle] (vi) at (0,2) {$v_i$};
        \draw (v1) edge[-] (v2)  (v3) edge[-] (vi) (vi) edge[-] (v1);
    \end{tikzpicture}
\end{figure}
Mögliche Kaskaden:
\begin{enumerate}
    \item die leere triviale Kaskade
    \item die Folge: $v_1 v_2$
\end{enumerate}
\end{frame}
\begin{frame}
\frametitle{Konfiguration}
\begin{defn}
Unter einer \emph{Konfiguration} verstehen wir eine Wahl von Teilgraphen $G_1,\dotsc,G_k$ wie in Satz 2 und genau eine Kaskade in jedem $G_i$ $\forall i=2,\dotsc,k$. Ein \emph{Kaskade Knoten} ist ein Knoten der zu einer Kaskade der Konfiguration gehört.
\end{defn}
\end{frame}
\begin{frame}
\frametitle{Verschiedene Definitionen}
\begin{defn}
Sei $w\in V(G_i)$ ein Kaskade Knoten. Wenn $w$ einen Nachbarn in $G_1$ hat, dann hat $w$ \emph{Rang} 1. Sonst in der Rang das kleinste $k\ge 2$, so dass ein Kaskade Knoten $w'\in V(G_j)$ existiert für ein $j=\lbrace 2,\dotsc,k\rbrace\setminus \lbrace i\rbrace$, so dass $w$ einen Nachbarn in $R(w')$ hat und $w'$ Rang $k-1$ hat. Wenn es keinen solchen Nachbarn gibt, ist der Rang nicht definiert. 
\end{defn}
\uncover<2>{\begin{defn}
Für ein $r\ge 1$ sei $\rho_r=\abs{R(w)}$ für einen Kaskade Knoten $w$ mit Rang $r$. Eine Konfiguration heißt \emph{gültig}, wenn jeder Kaskade Knoten einen wohldefinierten Rang hat und der Rang streng ansteigt innerhalb der Kaskade. Die triviale Konfiguration (nur leere Kaskaden) ist offensichtlich gültig.
\end{defn}}
\end{frame}
\begin{frame}
\frametitle{Beispiele}
\begin{figure}
\begin{tikzpicture}{
                \node[draw, circle] (v1) at (0,2) {$v_1$};
                \node[draw, circle] (v2) at (2,0) {$v_2$};
                \node[draw, circle] (v3) at (0,-2) {$v_3$};
                \node[draw, circle] (v4) at (-2,0) {$v_4$};
                \draw (v1) edge[-] (v2) (v2) edge[-] (v3) (v3) edge[-] (v4) (v4) edge[-] (v1);
                \draw[rotate=45] (0,1.35) ellipse (2.2 and 1);
                \draw[rotate=45] (0,-1.35) ellipse (2.2 and 1);
                \node (note) at (-2,2) {$G_1$};
                \node (note2) at (2,-2) {$G_2$};
            }
            \end{tikzpicture}
            \end{figure}
            \uncover<2>{Offensichtlich nur die leere Konfiguration (offensichtlich gültig)}
\end{frame}
\begin{frame}
\frametitle{Beispiele (2)}
\begin{figure}
\uncover<2>{
  \begin{itemize}
        \item $G_2$: $v_4 v_3$ (nicht gültig)
        \item $G_1$: leer (gültig)
    \end{itemize}}
 \begin{tikzpicture}{
                \node[draw, circle] (v1) at (0,2) {$v_2$};
                \node[draw, circle] (v2) at (2,0) {$v_1$};
                \node[draw, circle] (v3) at (0,-2) {$v_3$};
                \node[draw, circle] (v4) at (-2,0) {$v_4$};
                \draw (v1) edge[-] (v2) (v2) edge[-] (v3) (v3) edge[-] (v4) (v4) edge[-] (v1);
                \draw plot [smooth cycle] coordinates {(0,0) (0.8,-2) (0,-2.8) (-2.6,0) (0,2.8) (0.8,2.3)};
                \draw (v2) circle (0.7);
                \node (note) at (-2,2) {$G_2$};
                \node (note2) at (3.3,0) {$G_1$};
            }
            \end{tikzpicture}
            \end{figure}
          
\end{frame}
\begin{frame}
\frametitle{Letzte Definitionen}
\uncover<1->{\begin{defn}
Für $r\ge 1$ heißt eine gültige Konfiguration \emph{$r$-optimal}, wenn sie unter allen gültigen Konfigurationen $\rho_1$ maximiert unter der Bedingung, dass es $\rho_2$ maximiert usw. bis $\rho_r$. Wenn eine Konfiguration $r$-optimal ist für alle $r$, dann heißt sie einfach nur \emph{optimal}. \end{defn}}
\uncover<2>{
\begin{defn}
Zum Schluss definieren wir $S:=V(G)\setminus \bigcup G_i$. Das ist in dem Aufbau von Satz 2 nichtleer. Wir sagen eine \emph{Brücke} ist eine Kante mit einem Ende in $S$ und dem anderen Ende in einem Kaskade Knoten. In einer gültigen Konfiguration ist der \emph{Rang der Brücke} das Minimum der Ränge aller Kaskade Knoten $w$, wobei die Brücke ein Ende in $R(w)$ hat. 
\end{defn}}
\end{frame}
\begin{frame}
\frametitle{Ab jetzt beweisen wir}
\begin{lem}[Lemma 1]
Wenn es eine optimale Konfiguration gibt, die eine Brücke enthält, dann folgt Satz 2.
\end{lem}
\end{frame}
\begin{frame}
\frametitle{Beweis}
\begin{itemize}[<+->]
    \item Angenommen es gibt eine optimale Konfiguration mit Brücke
    \item Für $r\in\nn~\exists$ Brücke mit Rang $r$ (sei $r$ minimal)
    \item Endpunkte der Brücke $a\in S$, $b\in R(w),w\in G_i$, Rang $w=r$
    \begin{itemize}[<+->]
        \item Angenommen $w$ trennt $G_i$
        \item Alle Kaskade Knoten in $V(G_i)\setminus R(w)\setminus w$ haben Rang größer $r$
        \item Wähle aus diesen ein $u$, welches nicht-trennend ist
        \item neue gültige Konfiguration: $u\rightarrow S$ und $a\rightarrow G_i$ 
        \item lösche dazu Kaskade Knoten aus der obigen Menge
        \item Rang der betroffenen Knoten $>r$
        \item neue gültige Konfiguration und $R(w)$ ist größer 
        \item Widerspruch zur r-Optimalität
    \end{itemize}
\end{itemize}
\end{frame}
\begin{frame}
    \frametitle{Beispiel vorher}
    \begin{figure}
        \begin{tikzpicture}
            \node[draw, circle] (vi) at (-2,0) {$v_i$};
            \node[draw, circle] (z1) at (0,0) {$z_1$};
            \node[draw, circle] (z2) at (2,0) {$z_2$};
            \node[draw, circle] (w) at (4,0) {$w$};
            \node[draw, circle] (b) at (3,2) {$b$};
            \node[draw, circle] (a) at (6,4) {$a$};
            \node[draw, circle] (u) at (6,0) {$u$};
            \draw (vi) edge[-] (z1) (z1) edge[-] (z2) (z2) edge [-] (b) (b) edge[-] (w) (z2) edge[-] (w) (b) edge[-] (a) (w) edge[-] (u);
            \draw (2,0) ellipse (5 and 3);
            \node at (0,3) {$G_i$};\pause
            \draw[red] plot [smooth cycle] coordinates{(-1,0) (1.5,2) (4,2.5) (2.2,-1) (-1,-1)};
            \node[red] at (-1,-1.5) {$R(w)$};
        \end{tikzpicture}
    \end{figure}
\end{frame}
\begin{frame}
    \frametitle{Beispiel nachher}
     \begin{figure}
        \begin{tikzpicture}
        \node[draw, circle] (vi) at (-2,0) {$v_i$};
        \node[draw, circle] (z1) at (0,0) {$z_1$};
        \node[draw, circle] (z2) at (2,0) {$z_2$};
        \node[draw, circle] (w) at (4,0) {$w$};
        \node[draw, circle] (b) at (3,2) {$b$};
        \node[draw, circle] (a) at (6,0) {$a$};
        \node[draw, circle] (u) at (6,4) {$u$};
        \draw (vi) edge[-] (z1) (z1) edge[-] (z2) (z2) edge [-] (b) (b) edge[-] (w) (z2) edge[-] (w) (b) edge[-] (a) (w) edge[-] (u);
        \draw (2,0) ellipse (5 and 3);
        \node at (0,3) {$G_i$};\pause
        \draw[red] plot [smooth cycle] coordinates{(-1,0) (1.5,2) (4,2.5) (5,1) (7,0) (6,-1) (4,1) (2.2,-1) (-1,-1)};
        \node[red] at (-1,-1.5) {$R(w)$};
        \end{tikzpicture}
    \end{figure}
\end{frame}
\begin{frame}
    \frametitle{Weiter im Beweis}
     \begin{itemize}[<+->]
         \item Also trennt $w$ nicht $G_i$. Angenommen $r=1$.
         \item Dann hat $w$ einen Nachbarn in $G_1$
         \item Definiere $G_1'=G_1+w$ und $G_i':=(G_i+a)\setminus w$
         \item Bei allen anderen $j$ ist $G_j'=G_j$ und damit folgt der Schluss von Satz 2
         \item Sei also $r>1$. $w$ hat also einen Nachbarn in einem $R(w')$ mit Rang $w'=r-1$.
         \item Neue Konfiguration: $w\rightarrow S$ und $a\rightarrow G_i$. 
         \item Kaskaden: Kaskade vor $w$ abbrechen und bei undefiniertem Rang Knoten ausschließen
         \item Reservoirs vom Rang $\ge r$ beeinflusst aber immernoch $r-1$ optimal
         \item Es gibt jetzt eine Brücke von Rang $r-1$ (w zu seinem Nachbar in $R(w')$
         \item Widerspruch zur Minimalität von $r$ \qed
     \end{itemize}
\end{frame}
\begin{frame}
\frametitle{Noch ein Lemma}
\begin{lem}
Angenommen es gibt eine optimale Konfiguration mit einer Kante $ab$, so dass 
\begin{enumerate}
\item Entweder $a\in V(G_1)$ oder $a$ ist in einem Reservoir und
\item $b\in V(G_i)$ für ein $r\in\lbrace 2,\dotsc,k\rbrace,~b\neq v_i$ und $b$ ist nicht in einem Reservoir.
\end{enumerate}
Dann ist die Kaskade von $G_i$ nicht leer und $b$ ist der letzte Knoten dieser Kaskade.
\end{lem}
\end{frame}
\begin{frame}
\frametitle{Beweis}
siehe Ausarbeitung
\end{frame}
\begin{frame}
\frametitle{Beweis von Satz 2}
\begin{itemize}
\item Wir nehmen an: $G$ hat eine optimale Konfiguration ohne Brücke der aber Lemma 2 genügt.
\item Betrachte die Menge $M$, die folgende Knoten enthält:
\begin{itemize}
\item die letzten Knoten einer nicht-leeren Kaskade
\item $v_i$, falls $G_i$ nur die leere Kaskade besitzt
\end{itemize}
\item $M$ ist dann ein $k-1$-Schnitt von $G$, der $G_1$ und die Reservoirs vom Rest des Graphen (inkl. $S$) trennt.
\item Widerspruch zum $k-$Zusammenhang
\item Also sind die Voraussetzungen von Lemma 1 erfüllt.
\item Daraus folgt Satz 2
\end{itemize}
\end{frame}
\end{document}